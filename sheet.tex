\documentclass{article}
\usepackage{geometry}
\geometry{top=.5in,left=.5in,right=.5in,bottom=.5in}

\usepackage{physics}
\usepackage{amsthm}
\usepackage{lipsum}
\usepackage{amsmath}

\begin{document}
  \twocolumn
    \subsection*{General Facts}
      Suppose $\hat{A}$ is either Hermitan or unitary. Then  $\hat{A}$ admits a 
      spectral decomposition. That is 
      \[
         \hat{A} = \sum_i \lambda_i \ket{i}\bra{i}
      .\] 
      For some basis $\{\ket{i} \}$. For operators with continuious spectrum,
      this sum is replaced by an integral. 
      Hermitan operators have real eigenvalues and unitary operators 
      have eigenvalues which lie on the complex unit circle.
      If two operators $\hat{A}$ and  $\hat{B}$ commute, then it is possible 
      to find a basis  $\{\ket{i}\}$ where $ \ket{i}$ is a simultaneous eigenket
      of both $\hat{A}$ and  $\hat{B}$
      If $\hat{A}$ is Hermitan, then  $e^{-i \hat{A} \cdot a / \hbar}$
      is Unitary

      The projector from one basis $ \{\ket{a_i}\}$ to another basis 
      $\{\ket{b_i}\}$ is a unitary operator \[
        \hat{U} = \sum_i \ket{b_i}\bra{a_i}
      .\] 
        
    \[
      \sigma_x = \begin{pmatrix} 0 & 1 \\ 1 & 0 \end{pmatrix} \quad
      \sigma_y = \begin{pmatrix} 0 & -i \\ i & 0 \end{pmatrix} \quad
      \sigma_z = \begin{pmatrix} 1 & 0 \\ 0 & -1 \end{pmatrix}
    \]
 
    \[ 
      [\sigma_i,\sigma_j] = \delta_{a b}I + i \epsilon_{ijk}\,\,\sigma_k
    \]

    \[
      S_i = \frac{\hbar}{2}\sigma_i 
    \]

    \[
      \hat{p} = - i \hbar \frac{\partial}{\partial\,x} \quad
      \hat{x} = x \quad
      \hat{U} = e^{-i \hat{H} t / \hbar}
    \]
    
    \[
      [\hat{p},\hat{x}^n] = -i \,n \hbar \hat{x}^{n-1} \quad
      [\hat{x},\hat{p}^n] = i\,n  \hbar \hat{p}^{n-1} \quad
    \]
    \[
      [\hat{x},F(\hat{p})] = i \hbar \frac{\partial F}{\partial p} \quad
      [\hat{p},F(\hat{x})] =-i \hbar \frac{\partial F}{\partial x}
    \] 
    For some operator \(\hat{O}\) in the Schrodinger picture,
    the corresponding operator in the Heisenberg picture is 
    \(\hat{U}^\dagger\, \hat{O} \, \hat{U}\). And time dependence 
    in the Schrodinger picture is carried by \(\hat{U}\).
    

    \[
      i \hbar \frac{\partial}{\partial t} \ket{\varphi} = \hat{H} \ket{\varphi} 
      \iff \frac{d \hat{A}}{dt} = \frac{1}{i \hbar}[\hat{H},\hat{A}] + 
      \frac{\partial \hat{A}}{\partial t} 
    \]
    \subsection*{Harmonic Oscillator} 
      \[
        V = \frac{1}{2} m \omega^2 x^2 \quad 
        \hat{a} = \sqrt{\frac{m \omega}{2 \hbar}}( \hat{x} + \frac{i}{m \omega}\hat{p})
      \]

      \[
        \hat{x} = \sqrt{\frac{\hbar}{2 m \omega}}(\hat{a}^\dagger + \hat{a}) \quad
        \hat{p} = i \sqrt{\hbar m \omega /2 }(\hat{a}^\dagger - \hat{a}) \quad
        \hat{H} = \hbar \omega(\hat{a}^\dagger\hat{a} + \frac{1}{2})
      \]
  
      \[
        [\hat{a},\hat{a}^\dagger] = 1 \quad
        \hat{a} \ket{n} = \sqrt{n + 1} \ket{n+1} \quad
        \hat{a}^\dagger \ket{n} = \sqrt{n}\ket{n-1}
      \]
  
      \[
        \hat{a}(t) = e^{-i \omega t}\hat{a}(0) \quad 
        \hat{p}(t) = - m \omega \hat{x}(0) \sin(\omega t) + \hat{p}(0) \cos(\omega t) 
      \]
    
      \[
        \hat{x}(t) = \hat{x}(0) \cos(\omega t) + \frac{\hat{p}(0)}{m \omega }\sin(\omega t)
      \]
    \subsection*{Parity and Symmetry}    
      The parity operator $\hat{P}$ is defined by its action on the position 
      operator, that is  $\hat{P}^\dagger \hat{x} \hat{P} = - \hat{x}$. Therefore
      it is also the case that \[
        \hat{P}\ket{x} = \ket{-x}  \quad \hat{P}\ket{p} = \ket{-p}
      .\] 
      From this, we can see that $\hat{P}^{\dagger}\hat{p}\hat{P} = -\hat{p}$ 
      An operator $\hat{O}$ has is a symmetry of $\hat{A}$ if 
      $[\hat{A},\hat{O}]=0$ and $\hat{O}$ preserves probabilities in general.
      Symmetries must be unitary.
      In the position basis, even functions are of even parity and odd 
      functions are of odd parity.
      
      If $\hat{H}$ is a Hamiltonian which has a potential with periodicity  $a$
      Then 
      \[
        \hat{\tau}(a) = e^{-\frac{i a \cdot \hat{p} }{\hbar}}
      .\]  
      Is a symmetry of the Hamiltonian 
      If $\hat{U}$ is a symmetry of some hamiltonian generated by  $\hat{Q}$
      Then  $[\hat{H},\hat{Q}] = 0 \implies \frac{d\hat{Q}}{dt} = 0$

    \subsection*{Translation and Bloch's Theorem}
      \[
        \hat{\tau}(a) \ket{x} = \ket{x + a}
      .\]  
      Momentum is the generator of translation. 
      If $\hat{H}$ is a Hamiltonian with potential that has period $a$
      and some (potentially infinite) number of disconnected wells, we may 
      label the eigenstates of  $\hat H$ as  $ \ket{n,E}$ where  $n$ 
      corresponds to the localization of the state, and  $E$ is the eigenvalue
      of  $\hat H $ corresponding to  $\ket{n,E}$. We can find a linear 
      combination of these states, $\ket{\theta,E} = \sum_n e^{i n \theta}\ket{n,E}$
      \[
        \hat{\tau}(a) \ket{\theta,E} = e^{-i\theta}\ket{\theta,E}
      .\] 
      When the number of wells is finite, this quantizes $\theta$
      The \textbf{Tight Binding Approximation} is the assumption that 
      \[
        \bra{n,E}\hat{H}\ket{n+m,E}=0 \,\,:\,\, |m| > 1
      .\] 
     Bloch's Theorem says that, in such a system, \[
       \bra{x}\ket{\theta}= e^{i \theta x /a}u_k(x) \,\,:\,\, u_k(x + a) = u_k(x)
     .\]  
     The \textbf{Brillioun Zone} associated with a potential is 
     The set of physically distinct values of $k$ for which energy is defined 
     in terms of  $k$.

  \subsection*{Scattering and Wave mechanics}
    \[
      V(x) = 
      \begin{cases}
        0 & x < a_1 \\
        V(x) & a_1 \leq x \leq a_2 \\
        0 & x > a_2
      \end{cases} 
      \implies 
    \]
    \[
      \varphi(x) = 
      \begin{cases}
        A e^{i k x} + B e^{-i k x} & x < a_1 \\
        garbage & a_1 \leq x \leq a_2 \\
        F e^{i k x} + G e^{-i k x} & x > a_2
      \end{cases}
    .\] 
    \[
    T = \left|\frac{F}{A}\right|^2 \quad R = \left|\frac{B}{A}\right|^2
    .\]    
   The $S$ matrix is defined by the relation.
   \[
   \begin{pmatrix}
     F \\
     B
   \end{pmatrix}
   = 
   \begin{pmatrix}
     S_{11} & S_{12} \\
     S_{21} & S_{22} 
   \end{pmatrix}
   \begin{pmatrix}
     A \\
     G
   \end{pmatrix}
   .\] 
   Which is unitary.


  \subsection*{The WKB Approximation}
    \[
    \kappa(x) = \sqrt{\frac{2m}{\hbar^2}(V(x) - E)} \quad
    k(x) = \sqrt{\frac{2m}{\hbar^2} E - V(x)}
    .\]    
    \[
      \varphi(x) = 
      \begin{cases}
        \frac{1}{\sqrt{k(x)}} \exp[\pm i \int^x k(x)\,\,dx ]& E > V(x) \\
        \frac{1}{\sqrt{\kappa(x)}} \exp[\pm \int^x \kappa(x) \,\,dx] & E < V(x)
      \end{cases}
    .\] 
    If $\frac{d V}{dx}|_{x=a} > 0$ 
    \[
      \frac{A}{\sqrt{\kappa(x)}}\exp\left[ -\int_a^x \kappa(x')\,dx' \right] +
      \frac{B}{\sqrt{\kappa(x)}}\exp\left[\int_a^x \kappa(x)\,\,dx'\right] =
    \] 
    \[
      \frac{2A}{\sqrt{k(x)}}\cos\left[\int_x^a k(x')\,dx' - \frac{\pi}{4} \right] -
      \frac{B}{\sqrt{k(x)}}\sin\left[\int_x^a k(x')\,dx' - \frac{\pi}{4}\right]
    .\] 
    If the derivative at $a$ flips sign, you just flip all limits of inegration
    to get the correct expression.
    \subsubsection*{Trig Identities for WKB}
      \[
        \sin(\theta \pm \frac{ \pi}{2}) = \pm\cos(\theta) \quad
        \cos(\theta \pm \frac{\pi}{2}) = \mp\sin(\theta)
      .\] 
      \[
        \sin(\alpha \pm \beta) = \sin(\alpha)\cos(\beta) \pm \cos(\alpha)\sin(\beta)
      .\] 
      \[
        \cos(\alpha \pm \beta) = \cos(\alpha)\cos(\beta) \mp \sin(\alpha) \sin(\beta)
      .\] 
      \[
        2\cos(\theta)\cos(\varphi) = \cos(\theta - \varphi) + \cos(\theta + \varphi)
      .\] 
       \[
         2\sin(\theta)\sin(\varphi) = \cos(\theta - \varphi) - \cos(\theta + \varphi)
      .\] 
      \[
        2 \sin(\theta)\cos(\varphi) = \sin(\theta + \varphi) + \sin(\theta - \varphi)
      .\] 
      \[
        2\sin(\theta)\sin(\varphi) = \sin(\theta + \varphi) - \sin(\theta - \varphi)
      .\] 



\end{document}
